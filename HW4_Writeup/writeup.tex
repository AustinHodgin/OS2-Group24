\documentclass[onecolumn, draftclsnofoot,10pt, compsoc]{IEEEtran}
\usepackage{titling}
\usepackage{url}
\usepackage{enumitem}
\usepackage{geometry}
\geometry{textheight=9.25in, textwidth=6.75in} % 0.75" margins
\usepackage{hyperref}
\usepackage{listings}
\usepackage{cite}

\def\name{Group 26}


\title{Homework 3\\\large CS444 Spring 2018 }
\author{Austin Hodgin, Joshua Novak, Trent Vasquez}
\date{\today}

\begin{document}

	% Title page
	\begin{titlingpage}
		\maketitle
		\begin{abstract}
			\noindent Assignment 4 involved editing the current slob allocator for the linux yocto kernel to use the best
			fit algorithm to look at the difference between fragmentation rates between the first fit algorithm for slob which 
			is the default option for this kernel and the best fit algorithm that we implemented. 
		\end{abstract}
	\end{titlingpage}


	% Document body
	\section*{Design}
	For this assignment we looked online to find what the best fit algorithm is and how we can implement it into our system.
	After doing this research and learning what it is we decided that we are going to implement it by having it linearly scanning the blocks and 
	the pages looking for the chunk of memory that best fits what it is trying to store in it with as little wast as possible. This is very diffrent from 
	the default way it is done using slob. Slob does this by implementing the first fit model. This scans the blocks and pages and puts it in the 
	first place it fits with no regard to how large that block or page is.  
	
	\section*{Questions}
	\subsection*{1. What do you think the main point of this assignment is?}
	\noindent The main point of this assignment was to learn how to write system calls and to continue to learn 
	skills in regards kernel programing. We believe the main in point of this was to give us an understanding how
	to implement different storage algorithm and how they can affect the different levels fragmentation in memory from 
	the slob algorithm and the best fit algorithm. 

	\subsection*{2. How did you personally approach the problem? Design decisions, algorithm, etc.}
	\noindent The approach that our team went with for dealing with the problem is to make sure that the kernel was using 
	the default slob allocator. After that we modified the slob.c file to  include two different system calls and fixed any errors that 
	came from that. We then went on to rebuild the full kernel and wrote a kernel module to test to the  slop allocator by having the moduel 
	call kmalloc and kree. The kmalloc interface is built with the slob allocator and the simplest way to test is to
	calling kmalloc and kfree in a loop. After that we then wrote a program that would give us the fragmentation rates for both 
	the default and then try and implement the best fit algorithm and then test that with the same script. 


	\subsection*{3. How did you ensure your solution was correct? Testing details, for instance.}
	\noindent below are the steps and commands that we used to make sure our solution was correct as well as the 
	out put of the program that we wrote for testing the memory allocation for both first-fit and best-fit. 


	\begin{itemize}

	\item Source Environment:
		\begin{lstlisting}
		Source /scratch/files/environment-setup-i586-poky-linux.csh
		\end{lstlisting}
	
	\item Clean and build kernel: 
		\begin{lstlisting}
		cd linux-yocto-3.19
		make clean && make -j4 all
		\end{lstlisting}

	\item Run qemu using the command below to enable networking and file transfers this is all on one line:
		\begin{lstlisting}
		qemu-system-i386 -redir tcp:5524::22 -nographic 
		-kernel linux-yocto-3.19/arch/x86/boot/bzImage 
		-drive file=core-image-lsb-sdk-qemux86.ext4 
		-enable-kvm -usb -localtime --no-reboot 
		--append ``root=/dev/hda rwconsole=ttyS0 debug``
		\end{lstlisting}
\end{itemize}

	\noindent First Fit:
	\begin{itemize}
	\item 
		\begin{lstlisting}

		\end{lstlisting}
	\end{itemize}

	\nodindent Best Fit:

	\begin{itemize}
	\item 
		\begin{lstlisting}

		\end{lstlisting}
	\end{itemize}


	\includegraphics[width=\linewidth]{BF.png}
	\includegraphics[width=\linewidth]{FF.png}

	\subsection*{4. What did you learn?}
	This assignment taught us how to write system calls, understand the slob allocator as an interface as well as how memory 
	is allocated in general operations. After hours of research and trail and error we wer able to implement the best fit algorithm for the slob allocator that was the 
	base option. 
	

\begin{center}
\section*{Work Log}
\noindent 06/03/2018 Joshua worked on planing and overview of the assignment for 1-2 hours
\noindent 06/03/2018 Austin and Joshua worked on the main program and writeup for 4 hours 
\end{center}

\section*{Version control log}


\bibliography{ref}
\bibliographystyle{IEEEtran}

\end{document}
